\section{Vorwort}
\nomenclature[abbr]{ZfW}{Zauberfertigkeitswert}
\nomenclature[abbr]{MS}{Meisterschirm}
Bei jeder Tätigkeit oder jedem Zauber stehen einige Informationen am Ende zusammengefasst. Einige davon halte ich für selbsterklärend, für die anderen folgt eine Beschreibung.

Als \textbf{Mindest-ZfW} habe ich einen Wert gewählt, ab dem ich es für sinnvoll erachte mit diesem Spruch oder dieser Tätigkeit Geld verdienen zu wollen.

Der \textbf{mögliche Verdienst} leitet sich aus den Angaben im Meisterschirm ab, hier liegt das letzte Wort aber beim jeweiligen Meister (MS 33). Grundsätzlich sind diese Angaben als normaler Rahmen anzusehen. Zusatzeinnahmen durch Zeitdruck seitens des Auftraggebers oder die Möglichkeit ein Berufsgeheimnis nutzen zu können, bleiben unberücksichtigt. Des Weiteren liegt es im Ermessen des jeweiligen Magiers, ob eine Bezahlung erfolgen soll. Es könnte zum Beispiel auch eine anderweitige Gegenleistung vereinbart werden.

Unter \enquote{\textbf{Konkurrenz}} fallen mögliche lokale, regionale oder sogar überregionale Gruppierungen, denen die Tätigkeit des Magiers ein Dorn im Auge sein könnte. Der Meister kann diese Konkurrenz nutzen, um dem Magier Steine in den Weg zu legen.

Grundsätzlich kann der örtlichen Praioskirche die Ausübung magischer Tätigkeiten ein Dorn im Auge sein, aber auch örtlich verbreiteter Aberglaube kann schnell dazu führen, dass ein Magier aus dem Dorf getrieben wird.

Bei einigen Zaubern sind zusätzliche Anmerkungen für den Meister eingebaut als kleine Handreichung zur Reaktion der Umwelt auf den Magier.
