\section{Magische Hilfe für Profane}
\nomenclature[abbr]{LC}{Liber Cantiones (Hardcover)}
\nomenclature[abbr]{AsP}{Astralpunkt}
\nomenclature[abbr]{LeP}{Lebenspunkt}
\nomenclature[abbr]{H}{Heller}
\nomenclature[abbr]{S}{Silberling}
\nomenclature[abbr]{HAM}{Hallen arkaner Macht}
\nomenclature[abbr]{min.}{mindestens}
\nomenclature[abbr]{W6}{sechsseitiger Würfel}
\nomenclature[abbr]{WdA}{Wege der Alchimie}
Grundsätzlich steht es einem Magier frei jede magische Dienstleistung, die eine Akademie anbietet, auch auf Wanderschaft anzubieten. In Ermangelung eines Labors oder einer gut ausgestatteten Bibliothek stößt er dabei aber eher an Grenzen als an der Akademie. Die Kosten sollten in diesen Fällen reduziert werden.

Ich habe Methoden, die garantiert immer illegal sind, nicht aufgeführt. Deshalb fehlt zum Beispiel der Imperavi vollständig. Ebenso fehlen Methoden, die mit hoher Wahrscheinlichkeit erst erfahreneren Magiern zur Verfügung stehen wie etwa ein Transmutare Körperform.

\paragraph{Alchimie}
Alles, was mit dem Alchimiekoffer und den Sprüchen des Magiers möglich ist, kann dieser auch anbieten. Es sollte dem Magier aber klar sein, dass Verbrauchsmaterialien des Koffers auch wieder aufgefrischt werden müssen. Für eine \textbf{alchimistische Analyse} können um die \SI{7}{\D} verlangt werden, da für die \textbf{Zubereitung eines Trankes} ein Labor vonnöten sein kann, lässt sich dies nur dann durchführen, wenn ein entsprechendes Labor gestellt wird. Werden auch die Zutaten dafür gestellt, liegen die Kosten dafür je nach Schwierigkeit zwischen \SI{5}{\D} und \SI{50}{\D}.

\paragraph{Analys Arcanstruktur, Oculus Astralis, Odem Arcanum (LC 22, 196, 197/198)}
Die magische \textbf{Analyse von Artefakten} kann sehr viel Geld abwerfen, doch ist fraglich, ob ein wandernder Adept Gegenstände zur Analyse gebracht bekommt. Viel eher könnte ein Magier auf Durchreise sich durch die Anwendung eines Oculus oder der Sichtbereich-Variante des Odem einen \textbf{ganzen Haufen verschiedener Gegenstände auf einen Blick} anschauen. Für diese magische \enquote{Analyse} könnte er pro Gegenstand nur vergleichsweise wenig verlangen und so auch all jene dazu bewegen ihm Erbstücke und anderes vorbeizubringen.

Sollte einer der oberflächlichen Hellsichtzauber anschlagen, so bleiben dem Magier mehrere Möglichkeiten: er könnte bei genügend wertvollen Artefakten einfach das Weite suchen, mit dem Analys ein interessantes Artefakt ein wenig näher analysieren oder vorher mit dem Eigentümer sprechen. Der Magier könnte auch schlicht lügen der Gegenstand wäre von einem Dämon beseelt und der Magier wäre bereit sich für wenig Geld der Sache anzunehmen...
\begin{itemize}
	\item Dauer: 6 Sekunden bis 22,5 Sekunden für oberflächliche Analyse, Stunden und Tage für tiefergehende Analyse
	\item AsP-Kosten: 4+
	\item Mindest-ZfW: 5 (für oberflächliche Analyse), 10 (für tiefergehende Analyse)
	\item Möglicher Verdienst: einige H pro Gegenstand für oberflächliche Analyse, zig D für vollständige Analyse; \SI{1,5}{\D} für Erkennung von Magie, \SI{12}{\D} und mehr für Analyse (HAM 45)
	\item Hilfstalente: Sinnenschärfe
	\item Konkurrenz: Akademien, unter Umständen die Hesindekirche
\end{itemize}

\paragraph{Animatio Stummer Diener (LC 24)}
Für das besondere Flair kann der Animatio herangezogen werden: jede Art von Bewegung kann mit dem Animatio erschaffen und mit einem Auslöser verknüpft werden. Türen, die sich auf Befehl öffnen, Flaschen, die auf Befehl ein Glas füllen, tanzende Schwerter, die im Grunde aber ganz harmlos sind, oder tödliche Fallen, die sich nach dem Auslösen wieder selbst scharf machen. Der Fantasie sind keine Grenzen gesetzt, lediglich die Fähigkeiten des Magiers können hier einen Strich durch die Rechnung machen. Besonders geeignet für Adelige und andere Personen, die sich gern imposanter darstellen wollen als sie tatsächlich sind, sei es für die eigene Eitelkeit oder mit dem Ziel den Gegenüber zu imponieren wie zum Beispiel bei einem bevorstehenden Geschäftsabschluss.
\begin{itemize}
	\item Dauer: mindestens 42 Sekunden
	\item AsP-Kosten: 12+
	\item Mindest-ZfW: 11
	\item Möglicher Verdienst: \SI{10}{\D} und mehr
	\item Hilfstalente: keine
	\item Hilfszauber: Motoricus (Zurückholen eines geworfenen Gegenstands)
	\item Konkurrenz: Akademien
\end{itemize}

\paragraph{Applicatus Zauberspeicher (LC 25)}
Auf Reisen wird ein Magier kaum die Möglichkeit haben sich für die Erschaffung eines richtigen Artefaktes mittels Arcanovi zurückzuziehen. Aber für die Erschaffung temporärer Artefakte mittels Applicatus reicht es allemal und diese Dienstleistung lässt sich auch zahlender Kundschaft anbieten. Der Fantasie sind hier kaum Grenzen gesetzt und die möglichen Wirkungen hängen in erster Linie vom Magier ab. Neben Fallen eignet sich der Applicatus auch für einmalige Zusatzeffekte von Geschossen.
\begin{itemize}
	\item Dauer: 1 Minute
	\item AsP-Kosten: 7+
	\item Mindest-ZfW: 5
	\item Möglicher Verdienst: wenige bis etliche D; \SI{7}{\D}+ (HAM 44)
	\item Hilfstalente: keine
	\item Hilfszauber: keine
	\item Konkurrenz: Akademien
\end{itemize}
Unter Verwendung des Zusatzzeichens \enquote{Satinavs Zeichen} können die erschaffenen Effekte noch deutlich verlängert werden, verlangen dem Magier dafür aber auch einiges mehr ab.
\begin{itemize}
	\item Dauer: 1 Minute
	\item AsP-Kosten: min. 3W6 + Kosten des wirkenden Spruches
	\item Mindest-ZfW: 10
	\item Möglicher Verdienst: etliche D
	\item Hilfstalente: keine
	\item Hilfszauber: keine
	\item Konkurrenz: Akademien
\end{itemize}

\paragraph{Balsam Salabunde (LC 37)}
Die Kosten für eine Heilung mit dem Balsam sollten vom Zustand des Patienten abhängig gemacht werden: leichte Schrammen sind eben ein anderes Kaliber als ein am Rande des Todes stehender. Zudem ist die Bereitschaft für magische Heilung einfachster Verletzungen zu bezahlen, von Vorbehalten aufgrund von Aberglauben einmal abgesehen, eher gering. Bei Unfällen oder Turnieren ist die Wahrscheinlichkeit am höchsten zahlungswillige Kundschaft zu finden. Leider sind Patienten am Rande des Todes in der Regel nicht mehr in der Lage in ein Geschäft mit dem Magier einzuwilligen, sodass dieser sich auf die Dankbarkeit des Patienten verlassen muss. Aber Unfälle können in gewissem Rahmen forciert werden und das Opfer einer im Wald aufgestellten Falle wird doch den zufällig in der Nähe befindlichen Magier am wenigsten verdächtigen...
\begin{itemize}
	\item Dauer: 5 Minuten
	\item AsP-Kosten: 5+
	\item Mindest-ZfW: 7
	\item Möglicher Verdienst: einige S für kleinere Verletzungen, D, wenn zusätzlich Wunden hinzukommen, etliche D für Verstümmelungen und jenseits der \SI{50}{\D} für die Rettung eines Lebens von der Schwelle des Todes (aufgrund der permanenten Kosten); \SI{1}{\D}/LeP, pro Wunde Zuschlag von \SI{2}{\D} (HAM 45)
	\item Hilfstalente: Heilkunde Wunden, Kräuter sammeln
	\item Hilfszauber: Klarum Purum (Zum Stoppen einer Vergiftung)
	\item Konkurrenz: Akademien; Heiler aller Couleur, vor allem profane
	\item Meisterhinweise: Heilzauberei ist in manchen Gegenden verpönt. Die Landbevölkerung kennt am ehesten noch umherziehende Perainegeweihte und auch diese predigen eher die profane Heilkunde.
\end{itemize}

\paragraph{Beherrschung brechen (LC 41)}
Eine unter einem Beherrschungszauber stehende Person von diesem Einfluss zu befreien, kann sich durchaus lohnen -- zumindest, wenn das Opfer reich genug ist und euch auch glaubt, dass sie wirklich unter einem Beherrschungszauber stand.
\begin{itemize}
	\item Dauer: 1 Minute
	\item AsP-Kosten: 8+
	\item Mindest-ZfW: 10
	\item Möglicher Verdienst: etliche D; jenseits der \SI{50}{\D} (bei permanenten Kosten)
	\item Hilfstalente: keine
	\item Hilfszauber: Analys Arcanstruktur
	\item Konkurrenz: Hof- und Leibmagier (bei Adligen), Akademien
\end{itemize}

\paragraph{Claudibus Clavistibor (LC 58)}
Wer träumt nicht von einer Tür, die ihm und nur ihm aufmacht? Mit einem Claudibus ist dies möglich, wenn auch sehr teuer. Aber wer solche Wünsche hat, dürfte auch das nötige Kleingeld haben. Der Zauberer könnte natürlich auch immer sich selbst noch Zutritt gewähren, könnte schließlich mal nützlich sein...
\begin{itemize}
	\item Dauer: 4,5 Sekunden
	\item AsP-Kosten: 3+
	\item Mindest-ZfW: 11
	\item Möglicher Verdienst: etliche D bis jenseits der \SI{50}{\D} für einen permanenten Claudibus
	\item Hilfstalente: keine
	\item Hilfszauber: keine
	\item Konkurrenz: Akademien
	\item Meisterhinweise: Die Bereitschaft für diesen Zauber zu bezahlen muss im Allgemeinen als sehr gering eingestuft werden. Der Kunde würde es in aller Regel vorziehen seine Tür zu verstärken oder ein anständiges Zwergenschloss zu verbauen. Eher käme einer zusätzliche Zauberfalle oder ein Zauberzeichen zur Versiegelung in Betracht.
\end{itemize}

\paragraph{Custodosigil Diebesbann (LC 62)}
Ein Einbrecher oder Grabschänder, der einmal mit den feurigen Eigenheiten eines Custodosigils konfrontiert wurde, wird zukünftig anderswo sein Glück versuchen, so er nicht durch die Brandverletzungen schrecklich entstellt ist oder sogar stirbt.
\begin{itemize}
	\item Dauer: 1 Stunde
	\item AsP-Kosten: 5+
	\item Mindest-ZfW: 5
	\item Möglicher Verdienst: wenige D abhängig vom Zeitpunkt der nächsten Wintersonnenwende, jenseits der \SI{50}{\D} für permanente Varianten
	\item Hilfstalente: keine
	\item Hilfszauber: keine
	\item Konkurrenz: Akademien
\end{itemize}

\paragraph{Dämonenbann (LC 63)}
Wenn Magier für etwas bekannt sind, dann dafür, dass sie sich mit Dämonen auskennen... mit deren Bekämpfung, versteht sich -- auch wenn das gemeine Volk gern jedem Magier Dämonenbündelei nachsagt, so sind Magier diejenigen, die bei Auftreten dämonischer Phänomene gerufen werden. Und das kann sich ein entsprechend ausgebildeter Antimagier zunutze machen.
\begin{itemize}
	\item Dauer: 1 Minute
	\item AsP-Kosten: 10+
	\item Mindest-ZfW: 10
	\item Möglicher Verdienst: min. \SI{5}{\D} aufgrund der Gefahrenzulage, \SI{50}{\D} und mehr (bei permanenten Kosten)
	\item Hilfstalente: keine
	\item Hilfszauber: Pentagramma
	\item Konkurrenz: weiße und graue Akademien
	\item Meisterhinweise: Die meisten weißen und viele graue Akademien bieten die Austreibung von Dämonen kostenlos an. Ein Gardist, der mit horrenden Geldforderungen konfrontiert wird, sollte auch entsprechend überrascht reagieren, wenn sich eine Akademie in der Nähe befindet. (HAM 45)
\end{itemize}

\paragraph{Gefunden! (LC 96)}
Ein Zauber, der vor allem dann, wenn er dringend gebraucht wird, teuer werden kann. Aufgrund der Erschwernisse bei kleinen Gegenständen, die dem Magier auch nicht selbst bekannt sind, muss der Gefunden! aber meisterlich beherrscht werden.
\begin{itemize}
	\item Dauer: 30 Minuten
	\item AsP-Kosten: 17
	\item Mindest-ZfW: 14
	\item Möglicher Verdienst: etliche D
	\item Hilfstalente: Sinnenschärfe
	\item Hilfszauber: keine
	\item Konkurrenz: keine
\end{itemize}

\paragraph{Klarum Purum (LC 138)}
Ähnlich wie beim Balsam lässt sich hier bis zu einem gewissen Grad mit der Dringlichkeit der Behandlung Geld verdienen. Leider bleibt für Opfer des Purpurblitzes keine Zeit mehr zum Verhandeln, weshalb auch hier der Magier auf die Dankbarkeit des Patienten bauen muss.
\begin{itemize}
	\item Dauer: 10,5 Sekunden
	\item AsP-Kosten: 1+
	\item Mindest-ZfW: 10
	\item Möglicher Verdienst: wenige D für die Entfernung Alkohols bis jenseits der \SI{100}{\D} für Giftanschläge
	\item Hilfstalente: Heilkunde Gift
	\item Hilfszauber: Abvenenum
	\item Konkurrenz: Hofmagier, Heilkundige
\end{itemize}

\paragraph{Motoricus Geisteshand (LC 181/182)}
Mit dem Motoricus kann man sich allerlei Gegenstände besorgen, aber selten einmal wird jemand auf einen Magier zukommen und ihn bitten dies zu tun. Stattdessen ist es eher möglich, entsprechendes \enquote{Marketing} vorausgesetzt, dass der Magier gewissen Personen unsichtbare Hiebe erteilen soll. Wer würde dem Opfer solch eine Geschichte glauben? Und die Botschaft käme wahrscheinlich sogar sehr deutlich an.
\begin{itemize}
	\item Dauer: Minuten
	\item AsP-Kosten: 3+
	\item Mindest-ZfW: 7
	\item Möglicher Verdienst: einige S bis D
	\item Hilfstalente: keine
	\item Hilfszauber: Körperlose Reise (Verhindert es beim Zaubern entdeckt zu werden), Visibili (Verhindert es beim Zaubern entdeckt zu werden)
	\item Konkurrenz: profane Eintreiber
	\item Meisterhinweise: In der Regel dürfte selbst mit entsprechendem \enquote{Marketing} kaum jemand bereit sein einen Magier dafür zu bezahlen unsichtbare Hiebe zu verteilen.
\end{itemize}

\paragraph{Nekropathia (LC 188)}
Mit den Toten sprechen zu können ist in Aventurien tatsächlich möglich. Da der Spruch aber selten, komplex, teuer ist und noch dazu allgemein als lästerlich angesehen wird, sollte der Magier behutsam für diese Fähigkeit werben. Hat erst einmal jemand angebissen, so sollte er sich das fürstlich bezahlen lassen. Ideal, wenn auch moralisch mindestens fragwürdig, wäre es natürlich in der Nähe eines Boronsackers Ausschau nach Hinterbliebenen zu halten, anhand der Grabsteine und des Äußeren könnte man sogar auf die finanzielle Situation schließen. Und auch wenn kein Kontakt möglich sein sollte, kann der Magier ja auf Verfahren zurückgreifen, die nicht nur irdisch, sondern auch in Aventurien praktiziert werden, sprich: sich etwas ausdenken.
\begin{itemize}
	\item Dauer: mindestens 20 Minuten
	\item AsP-Kosten: 16+
	\item Mindest-ZfW: 7
	\item Möglicher Verdienst: etliche D
	\item Hilfstalente: Heilkunde Seele (Um zu wissen, was man dem Opfer erzählen sollte, wenn keine Verbindung zustandekommt)
	\item Hilfszauber: keine
	\item Konkurrenz: keine
	\item Meisterhinweise: Die Exhumierung einer Leiche bedeutet quasi immer auch eine Grabschändung. Sollte der Boronssegen auf der Leiche liegen, so kommt dazu noch ein Gottesfrevel. Als Magier sollte man sich also tunlichst nicht die Finger schmutzig machen, da dies auch ein geeignetes Druckmittel des Auftraggebers sein kann oder dieser plötzlich nichts mehr von dem Auftrag wissen will. In den Zwölfgöttern treuen Gebieten ist es quasi unmöglich mit diesem Zauber Geld zu verdienen. Es kann dennoch immer wieder zu Umständen kommen, die den Magier ohne Grabschändung und Gottesfrevel gestatten diesen seltenen Zauber anzuwenden.
\end{itemize}

\paragraph{Ruhe Körper, Ruhe Geist (LC 220)}
Eine ruhige und erholsame Nacht vor einer wichtigen Prüfung? Oder endlich einen Weg diese Schlafstörungen zu bekämpfen? Gut, dass ein Magier mit diesem Spruch vorbeischaut. Aber auch bei einem Heiler könnte ein solcher Magier dafür sorgen, dass die Patienten sich schneller besser fühlen.
\begin{itemize}
	\item Dauer: 30 Sekunden
	\item AsP-Kosten: 6+
	\item Mindest-ZfW: 8
	\item Möglicher Verdienst: einige S bis D
	\item Hilfstalente: keine
	\item Hilfszauber: Zauberzeichen \enquote{Seelenruhe} (Hält deutlich länger an)
	\item Konkurrenz: Heiler
\end{itemize}

\paragraph{Transversalis (LC 262)}
Eine der klassischen Dienstleistungen einer Akademie: der Transport von Personen und Lasten. Diese Dienstleistung ist teuer und kann böse enden. Daher sollte sich der Magier dies gut bezahlen lassen.
\begin{itemize}
	\item Dauer: 4,5 Sekunden
	\item AsP-Kosten: 11+ (Personen), 7+ (Lasten)
	\item Mindest-ZfW: 7 (Personen), 14 (Lasten)
	\item Möglicher Verdienst: etliche D; Personen: \SI{3}{\D} pro Meile, Lasten: \SI{2}{\D} pro Stein und 10 Meilen (HAM 45)
	\item Hilfstalente: keine
	\item Hilfszauber: keine
	\item Konkurrenz: Akademien
	\item Meisterhinweise: Die Erfahrung des lebensfeindlichen Limbus ist nichts für Jedermann und selbst Magier werden darauf vorbereitet und dazu ausgebildet, um nicht in Panik zu geraten. Definitiv nicht die komfortabelste Reise für jemanden, der nicht weiß, was ihn erwartet.
\end{itemize}

\paragraph{Zauberzeichen (WdA 136 -- 166)}
Das Anbringen von Zauberzeichen als Dienstleistung anzubieten eignet sich sehr gut für reisende Magier, denn so ein Zauberzeichen hält sich einige Zeit, der Markt ist also relativ bald gesättigt. Der genaue Preis kann sich vor allem an den nötigen Kosten orientieren, die wiederum von der Komplexität abhängen. Kosten im Bereich von \SI{1}{\D} pro AsP sollten für einfache Zauberzeichen benutzt werden, bei schwierigeren kann das recht schnell noch teurer werden. Kosten von \SI{4}{\D} pro AsP werden von Akademien für die Erstellung von Zauberzeichen veranschlagt, für die Aufladung die Hälfte (HAM 45).
