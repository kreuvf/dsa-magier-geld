\documentclass[11pt,twoside]{article}      % Specifies the document class
\usepackage{polyglossia}
\setdefaultlanguage[variant=german,spelling=new]{german}
\usepackage[autostyle=true,german=quotes]{csquotes}

\usepackage{geometry}
\geometry{a4paper, left=2cm, right=2cm, top=3cm, bottom=3cm}

\usepackage{siunitx}
\sisetup{
	exponent-product = \cdot,
	fraction-function = \sfrac,
	group-digits = integer,
	group-minimum-digits = 4,
	group-separator = \text{.},
	input-product = *,
	inter-unit-product ={}\cdot{},
	list-units = repeat,
	mode = text,
	multi-part-units = repeat,
	output-decimal-marker = \text{,},
	output-product = \cdot,
	qualifier-mode = subscript,
	quotient-mode = fraction,
	range-units = repeat,
	separate-uncertainty = true,
	table-figures-uncertainty = 1,
	table-figures-integer = 1,
	table-number-alignment = center-decimal-marker
}%
\DeclareSIUnit{\K}{K}
\DeclareSIUnit{\H}{H}
\DeclareSIUnit{\S}{S}
\DeclareSIUnit{\D}{D}
\DeclareSIUnit{\ZfPstern}{ZfP*}

\usepackage{import}
\usepackage{todonotes}
\usepackage[intoc]{nomencl}
\renewcommand{\nomname}{Abkürzungsverzeichnis}
\makenomenclature
\usepackage{titling}
\newcommand{\subtitle}[1]{%
  \posttitle{%
    \par\end{center}
    \begin{center}\large#1\end{center}
    \vskip0.5em}%
}

\usepackage{xfrac}
\usepackage{booktabs}
\usepackage{multirow}
\usepackage{xunicode}
\usepackage{fontspec}
\setmainfont[Mapping=tex-text]{Linux Libertine O}

\usepackage{xltxtra}
\usepackage{graphicx}
\usepackage{longtable}
\usepackage{tabularx}
\newcolumntype{R}{>{\raggedleft\arraybackslash}X}%

\usepackage{threeparttable}

\usepackage{pdfcprot}
\usepackage[grumpy]{gitinfo2}

% Taken from Schr4t
\tolerance 1414
\hbadness 1414
\emergencystretch 1.5em
\hfuzz 0.3pt
\widowpenalty 10000
\vfuzz \hfuzz
\raggedbottom
% /Taken from Schr4t

\usepackage{xcolor}
\definecolor{linkgreen}{rgb}{0.051,0.50,0.15}
\definecolor{uriblue}{rgb}{0.051,0.15,0.50}
\definecolor{isugrey}{rgb}{0.2,0.2,0.2}

\usepackage[]{hyperref}
\hypersetup{
	pdfborder = {0 0 1},
	pdfborderstyle = {/S/U/W 1},
	breaklinks = true,
	colorlinks = false,
	citebordercolor = linkgreen,
	linkbordercolor = linkgreen,
	urlbordercolor = uriblue,
	pdftitle = {Geld verdienen als Gildenmagier -- eine inoffizielle Spielhilfe auf Basis von Das Schwarze Auge 4.1},
	pdfauthor = {Steven Koenig / Isurandil, Matthias Petzold},
	pdfsubject = {Betrachtung der einem Gildenmagier zur Verfügung stehenden Einnahmequellen},
	pdfkeywords = {Version: \gitHash{}, date: \gitAuthorDate{}, branch: \gitBranch{}}
}
\urlstyle{rm}

% Redefinitions
\renewcommand{\nobreakspace}[1]{ #1}% The space used is U+00A0 NO-BREAK SPACE; This is a fix for the error "! LaTeX Error: Command \nobreakspace unavailable in encoding EU1."

\let\olditemize\itemize
\renewcommand{\itemize}{
	\olditemize
	\setlength{\topsep}{0pt}
	\setlength{\partopsep}{0pt}
	\setlength{\parskip}{0pt}
	\setlength{\itemsep}{3pt}
	\setlength{\parskip}{0pt}
	\setlength{\parsep}{0pt}
}

\let\oldenumerate\enumerate
\renewcommand{\enumerate}{
	\oldenumerate
	\setlength{\topsep}{0pt}
	\setlength{\partopsep}{0pt}
	\setlength{\parskip}{0pt}
	\setlength{\itemsep}{3pt}
	\setlength{\parskip}{0pt}
	\setlength{\parsep}{0pt}
}
