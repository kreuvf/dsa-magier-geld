\section{Hinweise}
Die hier aufgelisteten Methoden sind keinesfalls vollständig. Ich würde mich sehr über Einsendungen mit weiteren Vorschlägen oder auch Anmerkungen freuen.
\paragraph{Lizenz}
Dieser Artikel ist unter der \href{https://creativecommons.org/licenses/by-sa/3.0/de/}{CC-BY-SA-3.0-Lizenz} veröffentlicht. Es folgt eine rechtlich nicht bindende Zusammenfassung in allgemein verständlicher Sprache:
\begin{itemize}
	\item Der Artikel darf in jedem Format oder Medium vervielfältigt und weiterverbreitet werden.
	\item Der Artikel darf verändert oder als Grundlage für eigene Werke genutzt werden. Der Zweck ist dabei nicht von Belang. Auch eine kommerzielle Verwertung ist erlaubt.
	\item Sie müssen angemessene Urheber- und Rechteangaben machen, einen Link auf die Lizenz anfügen und deutlich machen, ob Veränderungen vorgenommen wurden. Diese Angaben dürfen in jeder angemessenen Art und Weise gemacht werden, allerdings nicht so, dass der Eindruck entsteht, der Lizenzgeber unterstütze gerade Sie oder Ihre Nutzung besonders.
	\item Verändern Sie diesen Artikel oder benutzen Sie ihn als Grundlage für eigene Werke, so dürfen Sie den veränderten Artikel oder ihr neues Werk nur unter denselben Bedingungen weitergeben, unter die Sie diesen Artikel nutzen dürfen.
	\item Sie dürfen keine zusätzlichen Klauseln oder technische Verfahren einsetzen, die anderen rechtlich irgendetwas untersagen, was die Lizenz erlaubt.
\end{itemize}

\paragraph{Version}
Dieses Dokument wird mit dem Versionsverwaltungsprogramm \href{https://git-scm.com/}{Git} verwaltet. Die Version dieses Dokuments lautet \enquote{\gitAbbrevHash{}}\footnote{Die vollständige Versionsnummer lautet: \gitHash{}.} vom \gitAuthorIsoDate{} von \gitAuthorName{} (\href{mailto:webmaster@kreuvf.de}{webmaster@kreuvf.de}), der verwendete Branch ist \gitBranch. Ein Klon des Repositorys befindet sich unter \url{https://github.com/kreuvf/dsa-magier-geld}.
