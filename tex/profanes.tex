\section{Profanes}
\nomenclature[abbr]{WdS}{Wege des Schwerts}
\nomenclature[abbr]{ZBA}{Zoo-Botanica Aventurica}
\nomenclature[abbr]{WdZ}{Wege der Zauberei}
\nomenclature[abbr]{WdH}{Wege der Helden}
Nicht nur Abgänger aus Akademien, die ein wenig Wert auf handwerkliche Ausbildung legen, wie etwa Nostria, können auf Basis profaner Talente Geld verdienen. Je nach dem in welche Richtung die sonstigen Interessen des Magiers gehen, kann er damit ebenfalls Geld verdienen. Ich habe an dieser Stelle vor allem jene Talente aufgeführt, die man für gewöhnlich auch bei Magiern antreffen kann.

\paragraph{Nahrung/Kräuter sammeln (WdS 190)}
Die Metatalente zum Sammeln essbarer Pflanzen/von Kräutern erlauben dem Magier seine Beute nicht nur selbst zu konsumieren, sondern auch zu verkaufen. Unverarbeitetes Wirselkraut kann man beispielsweise für \SI{1}{\D} verkaufen (ZBA 273), andere Kräuter sind noch weitaus mehr wert.

\paragraph{Lehren (WdS 23)}
Das Talent \enquote{Lehren} wird leider eindeutig als \enquote{nicht für Helden gedacht} markiert: \enquote{Da vermutlich nur wenige Helden sich ernsthaft für eine Lehrmeister-Tätigkeit interessieren, wird die Pflege dieses Talentes wohl hauptsächlich Meisterpersonen vorbehalten bleiben.} Dies ist sehr schade, da die Regeln zum Thema Lehren dadurch nicht gut durchdacht sind: dieses Talent ist erst ab einem Talentwert von 3 überhaupt benutzbar! Nichtsdestotrotz kann ein Magier dieses Talent steigern und entsprechend einsetzen, um all jene Dinge, die er gern lehren würde, auch weitergeben zu können. Ob es sich dabei um die Alphabetisierung der aventurischen Bevölkerung handelt oder um einen Gefährten, der unbedingt auch die Verbotenen Pforten nutzen können möchte, kann der Magier erst einmal lehren, muss er nur noch zahlungswillige Schüler finden.

Auf der Durchreise wären seltene, aber leicht zu lehrende (magische) Sonderfertigkeiten als Gastdozent an einer Akademie denkbar. Aber auch die Lehre eines bestimmten Zaubers ist prinzipiell möglich. Eine weitere Möglichkeit besteht gerade für Kampf- und Leibmagier: befindet sich eine Ausbildungsstätte für Krieger oder Offiziere in der Nähe, so könnte der Magier eine oder mehrere Sonderlehrstunden für die angehenden Absolventen abhalten und dabei verschiedene Dinge lehren:
\begin{itemize}
	\item Erkennung häufig eingesetzter Kampfzauber (Geste und Formel gängiger Sprüche wie Blitz, Duplicatus, Fulminictus, Horriphobus, Ignifaxius und Plumbumbarum)
	\item Auswirkung diverser Kampfzauber live vorführen (gegebenenfalls Einfluss eines freiwilligen Zielobjekts beachten)
	\item Kampf gegen Gildenmagier (Bann des Eisens, Randbedingungen Sicht, Geste, Formel und Konzentration stören, Magieresistenz erhöhen, Macht des Zauberstabs)
	\item Kampf gegen magische Wesen (Immunität gegenüber profanen Waffen, Gegnertypen und deren Eigenheiten)
\end{itemize}
Es bietet sich gerade bei Ausbildungsstätten von Kämpfern an statt einer Bezahlung Übung im Kampf mit Stäben durch einen der Meister zu erhalten.

\paragraph{Thesisniederschriften (WdZ 79/80)}
Jeder Magier lernt nicht nur wie man eine Thesis liest und aus ihr lernt, sondern auch wie man selbst welche verfasst. Der Magier braucht dazu einiges an Zeit, Pergament und Schreibzeug. Er muss sowohl gut im Thesiszeichnen (Spezialisierung auf Thesiszeichnen des Talents Malen/Zeichnen möglich) als auch in Magiekunde sein und den Spruch, den er auf Pergament bannen will, beherrschen. Eine Qualität von 12 müsste über drei Tage verteilt erstellt werden, wobei an jedem Tag acht Stunden für die Thesisniederlegung aufgewendet werden müssten.

Die fertige Thesis kann dann verkauft werden. Es ist auch möglich, dass man als Spieler eine Thesis als Bezahlung für eine Meisterperson anbietet. Wer sagt denn auch, dass immer nur die Meisterpersonen im Besitz toller Dinge sein sollen, die die Spieler wollen, und es nicht mal umgekehrt laufen kann? Der Magier sollte aber gut darüber Bescheid wissen, welche Sprüche er problemlos weitergeben kann und welche als \enquote{Akademiegeheimnis} gelten wie etwa der \enquote{Blick in die Vergangenheit} (WdH 179).

Die Kosten für eine Thesis der Qualität 12 können nur abgeschätzt werden und sollten von der Verbreitung abhängen. Im Liber Cantiones finden sich einige Kosten für Thesisabschriften, die ich in Tabelle~\ref{tab:theses:kosten1} zusammengetragen habe. Leider sind dort keine Qualitätsangaben dabei. Vermutlich sind all das eher durchschnittliche Abschriften (Qualität: 7 bis 10). Alternativ kann die Formel aus HAM 45 verwendet werden:
	\begin{equation}
		K = (8 - V) \cdot C \cdot 50
	\end{equation}
Hierin steht $K$ für die Kosten in Silberlingen, $V$ für die Verfügbarkeit unter Gildenmagiern und $C$ für die Komplexität des Zaubers. Die entsprechenden Kosten nach dieser Formel sind ebenfalls in der Tabelle hinterlegt.
\begin{table}
	\centering
	\caption[Kosten für Thesisabschriften (Literaturwerte)]{Zusammenstellung der in LC und WdH angegebenen Kosten für Thesisabschriften mit Gegenüberstellung der Kosten nach HAM 45\label{tab:theses:kosten1}}
	\begin{tabular}{lrrrr}
		\toprule
		Spruch & Kosten (Quelle) & Kosten (HAM 45) & Verbreitung & Quelle\\
		\midrule
		Aerogelo & \SI{250}{\D} -- \SI{300}{\D} & \SI{120}{\D} & 2 & WdH 184, WdZ 277, LC 20\\
		Brenne, toter Stoff! & \SI{50}{\D} & \SI{75}{\D} & 3 & LC 51\\
		Caldofrigo Heiß und Kalt & \SI{250}{\D} & \SI{100}{\D} & 4 & WdZ 277\\
		Eigne Ängste quälen dich! & \SI{50}{\D} & \SI{90}{\D} & 2 & LC 74\\
		Erinnerung verlasse dich! & \SI{50}{\D} & \SI{100}{\D} & 3 & LC 82\\
		Favilludo Funkentanz & \SI{100}{\D} & \SI{15}{\D} & 5 & LC 85\\
		Hartes schmelze! & \SI{20}{\D} & \SI{75}{\D} & 3 & WdH 184, WdZ 277\\
		Pfeil der Luft & \SI{333}{\D} & \SI{75}{\D} & 3 & LC 209\\
		Transmutare Körperform & \SI{250}{\D} und mehr & \SI{120}{\D} & 2 & LC 261\\
		Weiches erstarre! & \SI{50}{\D} & \SI{75}{\D} & 3 & LC 278\\
		Zauberklinge Geisterspeer & \SI{144}{\D} & \SI{100}{\D} & 3 & LC 292\\
		\bottomrule
	\end{tabular}
\end{table}
Aus den konkret angegebenen Beispielen leite ich mir folgende Werte für eine Thesis der Qualität 7 beziehungsweise 12 für folgende Sprüche ab (je zwei Beispiele pro Verbreitung). Die Kosten nach HAM 45 stellen die Extremwerte für Komplexität A und Komplexität F dar. In Tabelle~\ref{tab:theses:kosten2} befinden sich die Kosten nach einer eigenen Näherung, die auf den in Tabelle~\ref{tab:theses:kosten1} gefundenen Werten basiert, und berechnet nach HAM 45.
\begin{table}
	\centering
	\caption[Kosten für Thesisabschriften (berechnet)]{Kosten für Thesisabschriften berechnet nach eigener Näherung und HAM 45. Zahlen hinter einem \enquote{Q} stehen für die Qualität. \enquote{Q7} steht daher für Qualität 7.\label{tab:theses:kosten2}}
	\begin{tabular}{lrrrr}
		\toprule
Verbreitung & Thesis (Q7) & Thesis (Q12) & Kosten (HAM 45) & Beispiele\\
		\midrule
		7 & \SI{5}{\D} & \SI{20}{\D} & \SI{5}{\D}/\SI{30}{\D} & Balsam, Flim Flam\\
		6 & \SI{8}{\D} & \SI{30}{\D} & \SI{10}{\D}/\SI{60}{\D} & Gardianum, Klarum Purum\\
		5 & \SI{12}{\D} & \SI{50}{\D} & \SI{15}{\D}/\SI{90}{\D} & Fulminictus, Invocatio Maior\\
		4 & \SI{30}{\D} & \SI{120}{\D} & \SI{20}{\D}/\SI{120}{\D} & Karnifilo, Skelettarius\\
		3 & \SI{60}{\D} & \SI{300}{\D} & \SI{25}{\D}/\SI{150}{\D} & Ecliptifactus, Transversalis\\
		2 & \SI{100}{\D} & \SI{500}{\D} & \SI{30}{\D}/\SI{180}{\D} & Tempus Stasis, Traumgestalt\\
		1 & \SI{900}{\D} & \SI{5000}{\D} & \SI{35}{\D}/\SI{210}{\D} & Immortalis, Infinitum\\
		\bottomrule
	\end{tabular}
\end{table}
%<!-- LibreOffice gibt dafür dann folgende Gleichungen als Näherung aus, R² jeweils 98+%:
%x: Verbreitung unter Gildenmagiern in gildenmagischer Repräsentation
%y: Kosten einer Thesisabschrift in D
%Qualität 7: 840 * x ^ (-2,6)
%Qualität 12: 4700 * x ^ (-2,8)
%Daraus könnte man sicherlich eine noch allgemeinere Formel bauen, die von den beiden Variablen Verbreitung und Qualität abhängt, aber dazu habe ich jetzt keine Lust :P
%-->
Ein Magier sollte also mit ein wenig Aufwand in der Lage sein eine Thesis zu schreiben, die ihm pro Stück locker \SI{20}{\D} einbringt.

\paragraph{Übersetzungen}
Ein weiteres mögliches Betätigungsfeld für Magier ist die Übersetzung. Neben der jeweiligen Lehrsprache kann jeder Magier mindestens noch eine weitere Sprache. Handelt es sich dabei sogar auch um eine allgemein gebräuchliche Sprache, könnte der Magier Übersetzungsdienste anbieten. Leider konnte ich nirgends Anhaltspunkte zum regeltechnischen Ablauf einer Übersetzung finden, geschweige denn zum erzielbaren Preis.

\paragraph{Leibmagiertätigkeit}
Neben direkt zum Leibmagier ausgebildeten Magiern kann auch ein sich für entsprechend fähig haltender Magier versuchen durch ein entsprechendes Angebot Geld zu erhalten. Diese Möglichkeit bietet sich andererseits aber auch sehr gut zum kostenlosen Reisen an: dafür, dass der Magier auf der Reise durchgefüttert wird, muss dieser sich nach Kräften um die Sicherheit des Schiffes oder der Karawane kümmern, wenn es gefährlich werden sollte. Der Preis sollte bei irgendwas um \SI{1}{\D} pro Tag liegen, wobei es an Spieler und Meister liegt besondere Umstände geltend zu machen, die den Preis verändern. HAM 45 listet als Grundpreis für magische Leibwächter einen Stundensatz von \SI{1}{\D} und einen Tagessatz von \SI{5}{\D}, wobei zusätzliche Zuschläge nach ausgegebenen AsP und ein Gefahrenzuschlag geltend gemacht werden können. Dabei ist zu beachten, dass die Kosten aus HAM 45 für von einer Akademie gestellten Leibwächter gelten.