\section{Handwerksmagie -- Magische Konkurrenz für Profane}
\nomenclature[abbr]{ZfP*}{Zauberfertigkeitspunkte nach der Probe}
\nomenclature[abbr]{AP}{Abenteuerpunkte}
\nomenclature[abbr]{K}{Kreuzer}
Auch wenn magisch unbegabte oder zumindest magisch nicht ausgebildete Handwerker dies nicht gern sehen, so hat die Gildenmagie doch den ein oder anderen Spruch im Gepäck, der in direkter Konkurrenz zum Handwerk steht. Es sollte aber auch jedem Magier bekannt sein, dass Handwerksmagie von \enquote{echten} Magiern eher abwertend betrachtet wird. Da Handwerksmagie immer auch in \textit{Konkurrenz zum Handwerk} steht, lassen sich in der Regel \textit{keine so hohen Preise} erzielen wie bei pur magischen Dienstleistungen, weshalb die Formel Zauberfertigkeitswert mal Astralpunktkosten mal Komplexität des Zaubers in Heller (MS 33) hier keine Anwendung findet. Eine magische Dienstleistung hat dennoch immer einen gewissen Magiebonus.

Nicht explizit aufgeführt bei Sprüchen, deren Wirkung nicht dauerhaft ist (zum Beispiel Accuratum), ist die Anwendung eines Protectionis, um Antimagier an der Entzauberung zu hindern. Diesen Zusatzdienst sollte der Magier sich wie eine magische Dienstleistung bezahlen lassen.

\paragraph{Accuratum Zaubernadel (LC 12)}
In nur einer Minute kann ein Gildenmagier Löcher in Socken, Handschuhen oder sogar Praiostagsanzügen dauerhaft flicken. Selbst die Erzeugung fertiger Kleidung aus rohen Stoffbahnen ist möglich. Besitzt der Magier noch das Talent \enquote{Schneidern}, kann er dies zur Unterstützung seiner Magie heranziehen. Zwar anfällig für Antimagie, aber für vorübergehende Kleidung nützlich ist eine Variante, die es erlaubt selbst die Art des Stoffes zu verändern. Kombiniert mit der hohen Geschwindigkeit, mit der ein Gildenmagier entsprechende Kleidung herrichten kann, sind auch vernünftige Löhne möglich. Da bereits \SI{10}{\ZfPstern} für ein meisterliches Ergebnis stehen, lässt sich für das einfache Volk bereits mit \SI{5}{\ZfPstern} ein zufriedenstellendes Ergebnis erzielen. Ein Aureolus kann die Kleidung wertvoller erscheinen lassen, ein Sapefacta erlaubt die einfache Reinigung der Kleidung vor der Bearbeitung. Besonders die Eigenschaft ohne Investition permanenter Astralpunkte dauerhafte Kleidungsstücke zu erschaffen, erlauben es mit diesem Spruch leicht an Geld zu gelangen ohne ständig AP für den Rückkauf permanenter Astralpunkte aufwenden zu müssen.
\begin{itemize}
	\item Dauer: 1 Minute bis 75 Minuten. Für schwierige Stücke, eine gedanklich ausgeführte Schneidern-Probe oder die Webstuhl-Variante muss zusätzliche Zeit einkalkuliert werden.
	\item AsP-Kosten: 7+
	\item Mindest-ZfW: 5
	\item Möglicher Verdienst: wenige K für gestopfte Socken bis hin zu mehreren D für meisterliche Arbeiten für Adlige
	\item Hilfstalente: Schneidern
	\item Hilfszauber: Aureolus, Sapefacta
	\item Konkurrenz: Schneider
	\item Meisterhinweise: Auch wenn ein ZfW von 5 für das gemeine Volk ausreicht, kommen die Auftraggeber viel eher aus den Reihen der Bürger oder Adligen, weshalb eher ein ZfW von 10 vorhanden sein sollte.
\end{itemize}

\paragraph{Aureolus Güldenglanz (LC 33)}
Dieser Spruch eignet sich nicht nur für den offensichtlichen Zweck den Schein von Reichtum zu erzeugen, sondern auch eher wertlose Gegenstände wertvoll erscheinen zu lassen. Werden diese dann verkauft und verlieren ihren Glanz am nächsten Morgen, sollte der Magier sich aber nicht in der Nähe des Käufers aufhalten. Aber es hat ja niemand gesagt, dass der Magier dieses verkaufen muss. Ein gewiefter Magier könnte hier sogar doppelt abkassieren: zuerst wird das wertlose Objekt des Betrügers vergoldet, dann \enquote{analysiert} der Magier am nächsten Morgen das Objekt, um die erschreckende Wahrheit herauszufinden: hier hat jemand mit magischen Methoden betrogen. Und der Magier kann sogar erkennen, wer der Übeltäter ist...
\begin{itemize}
	\item Dauer: 7,5 Sekunden
	\item AsP-Kosten: 3+
	\item Mindest-ZfW: 3
	\item Möglicher Verdienst: wenige H für handgroße Gegenstände, mehrere S für Einrichtungsgegenstände, D für ganze Zimmer
	\item Hilfstalente: keine
	\item Hilfszauber: keine
	\item Konkurrenz: Juweliere, Goldschmiede (Feinarbeiten)
\end{itemize}

\paragraph{Delicioso Gaumenschmaus (LC 64)}
Ein Gastwirt, dessen Geschäft schlecht läuft, könnte vom Magier ein Angebot unterbreitet bekommen, das dieser nicht ablehnen kann: für eine Beteiligung an den Einnahmen kann der Magier dafür sorgen, dass jeder im Dorf sein Essen loben wird. Noch einfacher ginge es natürlich, wenn der Magier die Speisen komplett selbst herstellt und verkauft, was in der Regel aber mangels Küche problematisch sein dürfte. Es ist natürlich auch denkbar, dass ein Magier die Variante \enquote{Widerwärtiger Geschmack} auf das Essen eines Konkurrenten zaubert und so für neue Kunden sorgt.

Abweichend von der Beschreibung des Spruches könnte man auch hier eine gedanklich ausgeführte Kochen-Probe anrechnen, um die Probe zu erleichtern, da ein begabter Koch sehr viel besser weiß, welche Zutaten welchen Geschmack erzeugen. Aufgrund der kurzen Haltbarkeit des Spruches, kann der Magier leider nicht einfach am Morgen den Spruch auf einen Kessel Suppe sprechen, die den gesamten Tag über verteilt wird.
\begin{itemize}
	\item Dauer: 15 Sekunden bis mehrere Stunden. Kann der Magier kochen und möchte sein Gericht so unwiderstehlich gut machen, dass er fast alles dafür verlangen kann, kommt die Kochzeit noch hinzu.
	\item AsP-Kosten: 3+
	\item Mindest-ZfW: 6
	\item Möglicher Verdienst: wenige H bis D
	\item Hilfstalente: Kochen, Schleichen (Die Reichweite des Spruches erlaubt es nicht das Essen der Konkurrenz aus großer Entfernung zu verzaubern)
	\item Hilfszauber: Körperlose Reise, Visibili (Verhindern es beim Verzaubern des Essens der Konkurrenz entdeckt zu werden)
	\item Konkurrenz: Gastwirte, Bäcker, Köche
	\item Meisterhinweise: Bei mehreren Gästen wird das Vorhaben schnell sehr teuer und der Koch fliegt auf. Wenn der Koch nicht kochen kann, hat er ab Abreise des Magiers schnell keine Kunden mehr. Dieser Zauber ist daher eher als kleine Aufwertung gedacht, weil der Koch einen seltenen, aber edlen Gast bekommt und befürchtet diesem nicht gerecht zu werden.
\end{itemize}

\paragraph{Motoricus, Variante \enquote{Rote und Weiße Kamele} (LC 181/182)}
Wie von Geisterhand bewegte Spielfiguren können auf den ein oder anderen eine große Faszination ausüben und so könnte ein Magier durch vergleichsweise geringen Aufwand einen Satz Spielfiguren verzaubern und für das Spiel damit Geld verlangen. Leider muss der Magier die gesamte Zeit über dabei sein, was den Stundenlohn enorm schmälert. Es ist auch möglich, dass der Magier selbst eine Partie gegen einen Kandidaten spielt und der Sieger einen Gewinn erhält. Die Gefährten könnten unterdessen Wetteinsätze vom Publikum einsammeln.

\begin{itemize}
	\item Dauer: 7,5 Sekunden + Spielzeit
	\item AsP-Kosten: 9 AsP
	\item Mindest-ZfW: 7
	\item Möglicher Verdienst: mehrere S bei Adligen oder Spielsüchtigen
	\item Hilfstalente: Brettspiel (Kennt der Magier die Regeln des Spiels, kann er die Figuren flüssiger bewegen. Das reduziert die Zeit pro Spiel und erlaubt es andere Teilnehmer zahlen zu lassen)
	\item Hilfszauber: Aureolus
	\item Konkurrenz: keine
\end{itemize}

\paragraph{Pectetondo Zauberhaar (LC 203)}
Frisur zerzaust? Bart falsch geschnitten? Alles kein Problem für den Pectetondo -- und entsprechend ausgebildete Magier. Auch sind gezielte \enquote{Anschläge} möglich, um eine wichtige Person der Lächerlichkeit preiszugeben. Dass die Wirkung nicht wie bei einem Barbier nur langsam verfliegt, ist Vor- und Nachteil zugleich. Besonders bei gesellschaftlichen Anlässen wie einem Bankett geeignet.

\begin{itemize}
	\item Dauer: 9 Sekunden
	\item AsP-Kosten: 4+
	\item Mindest-ZfW: 3
	\item Möglicher Verdienst: einige H bis hin zu D für besonders extravagante Kreationen
	\item Hilfstalente: Schleichen (Die Reichweite des Spruches erlaubt es nicht das Ziel ohne Weiteres aus großer Entfernung zu verzaubern)
	\item Hilfszauber: Körperlose Reise, Visibili (Verhindern es beim Verzaubern des Essens der Konkurrenz entdeckt zu werden)
	\item Konkurrenz: Barbiere, Bader
\end{itemize}

\paragraph{Sapefacta Zauberschwamm (LC 223)}
Und da fällt die Braut auch schon in eine riesige Pfütze voll braunem Schlamm. Sollte der Magier dies beobachten, könnte er ihr für wenig Geld anbieten die Sauerei zu entfernen, ohne dass sie sich ausziehen müsste. Alternativ kann der Magier auch versuchen seine Fähigkeiten als magische Waschmaschine anders zu vermarkten. Das Wäschewaschen gehört schließlich auch in Aventurien zum Alltag und für nur kleines Geld kann man sich dieser lästigen Pflicht entledigen. In der Variante \enquote{Läusekamm} kann der Magier sogar dafür sorgen störende Kleintiere zu beseitigen.

\begin{itemize}
	\item Dauer: 75 Sekunden
	\item AsP-Kosten: 5
	\item Mindest-ZfW: 7
	\item Möglicher Verdienst: wenige K bis hin zu mehreren S
	\item Hilfstalente: keine
	\item Hilfszauber: keine
	\item Konkurrenz: keine
\end{itemize}

\paragraph{Weihrauchwolke Wohlgeruch (LC 279)}
Je nach dem, ob der Wohlgeruch nur einige Stunden oder sogar Tage wirken soll, kann hier von ein wenig bis viel Geld verdient werden. Da es in gewissen Schichten nicht üblich ist sich mit Düften zu umgeben, ist davon auszugehen, dass man mit einer Weihrauchwolke in ländlichen Gegenden kein Geld verdienen wird, sehr wohl aber in Städten.

\begin{itemize}
	\item Dauer: 7,5 Sekunden
	\item AsP-Kosten: 6
	\item Mindest-ZfW: 6
	\item Möglicher Verdienst: einige S bis D
	\item Hilfstalente: keine
	\item Hilfszauber: keine
	\item Konkurrenz: Alchimisten (Parfümeure)
\end{itemize}
